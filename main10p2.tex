\documentclass[conference]{IEEEtran}
\IEEEoverridecommandlockouts
% The preceding line is only needed to identify funding in the first footnote. If that is unneeded, please comment it out.
\usepackage{cite}
\usepackage{amsmath,amssymb,amsfonts}
\usepackage{algorithmic}
\usepackage{graphicx}
\usepackage{textcomp}
\usepackage{comment}
\usepackage{xcolor}
\def\BibTeX{{\rm B\kern-.05em{\sc i\kern-.025em b}\kern-.08em
    T\kern-.1667em\lower.7ex\hbox{E}\kern-.125emX}}

\begin{document}

\title{Theory HW 1 Q10 Part 2}

\author{\IEEEauthorblockN{Team 12}
\IEEEauthorblockA{\textit{Department of Computer Science} \\
\textit{University of Houston}\\}
}

\maketitle

\section{Question}
%
Optimized recursion on overlapping subproblems:
Show a step by step example aligning two sequences of length 10 combining 5 symbols with
optimized recursion (dynamic prog.) algorithm varying the gap score and mismatch scores.




\section{Pseudo-code}

\textbf{SimScoreDP}
\newline
Input: sequences s and t of lengths m and n
\newline
Output: sim(s, t)

\begingroup
\renewcommand\labelenumi{\theenumi:}
\begin{enumerate}
\item \textbf{function} SimScoreDP(s, t)  \label{item:1}
\item for i = 0 to m do \label{item:2}
\item a[i, 0] = -i  \label{item:3}
\item for j = 0 to n do \label{item:4}
\item a[0,j] = -j \label{item:4}
\item for i = 1 to m do \label{item:4}
\item \{ \label{item:4}
\item for j = 1 to n do \label{item:4}
\item \{ \label{item:4}
\item both = a[i - 1, j - 1] + score(s[i], t[j]) \label{item:4}
\item left = a[i, j - 1] + score(-, t[j]) \label{item:4}
\item right = a[i - 1, j] + score(s[i], -) \label{item:4}
\item a[i, j] = max(both, left, right) \label{item:4}
\item \} \label{item:4}
\item \} \label{item:4}
\item return a[m, n] \label{item:4}
\end{enumerate}
\endgroup

\section{Algorithm}
\begin{verbatim}
{ Function implementing dynamic programming 
to compute similarity score }
function SimScoreDP(s, t: string): integer;
var
  i, j: integer;
begin
  m := Length(s);
  n := Length(t);

  { Base case initialization }
  for i := 0 to m do
    a[i, 0] := -i; { Deletion penalty }

  for j := 0 to n do
    a[0, j] := -j; { Insertion penalty }

  { Fill DP table using the given recurrence }
  for i := 1 to m do
  begin
    for j := 1 to n do
    begin
      a[i, j] := Max(
        { Diagonal (match/mismatch) }
        a[i - 1, j - 1] + Score(s[i], t[j]),  
        { Left (insertion) }
        a[i, j - 1] + Score('-', t[j]),     
        { Up (deletion) }
        a[i - 1, j] + Score(s[i], '-')        
      );
    end;
  end;

  { Return final similarity score }
  SimScoreDP := a[m, n];
end;
\end{verbatim}


\section{Input examples}

\begin{verbatim}
Input
s = AABCCBAABC
t = ACCBBAABCC
Output
AABCC-BAAB-C
-A-CCBBAABCC
Score: 12

Input
s = ABCDEABCDE
t = EDCBAEDCBA
Output
ABCDEA-BCDE
EDC-BAEDCBA
Score: -2

Input
s = AAABBBCCDD
t = BBBAAACCDD
Output
AAABBB---CCDD
---BBBAAACCDD
Score: 8

\end{verbatim}

\section{Time Complexity}
When implimented in this way, the time complexity of this algorithm is theta($n^2$)

\section{AI Tools Used}
ChatGPT v3.5 was used to generate the Pascal code. Algorithms used come from Pandurangan 2022.
\end{document}