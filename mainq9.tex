\documentclass[conference]{IEEEtran}
\IEEEoverridecommandlockouts
% The preceding line is only needed to identify funding in the first footnote. If that is unneeded, please comment it out.
\usepackage{cite}
\usepackage{amsmath,amssymb,amsfonts}
\usepackage{algorithmic}
\usepackage{graphicx}
\usepackage{textcomp}
\usepackage{comment}
\usepackage{xcolor}
\def\BibTeX{{\rm B\kern-.05em{\sc i\kern-.025em b}\kern-.08em
    T\kern-.1667em\lower.7ex\hbox{E}\kern-.125emX}}

\begin{document}

\title{Theory HW 1 Q9}

\author{\IEEEauthorblockN{Team 12}
\IEEEauthorblockA{\textit{Department of Computer Science} \\
\textit{University of Houston}\\}
}

\maketitle

\section{Question}
%
Show an example of MCM with n = 8 matrices where DP produces about the same result with
a naive left-right multiplication.
Show an example of MCM with n = 8 matrices where DP produces signficantly better performance than a naive left-right multiplication.
Rewrite the MCM algortihm with recursion, instead of loops



\section{Pseudo-code}

\textbf{NaiveMCM}
\newline
Input: A sequence of matrix dimensions: p = p0..pn
Output: The optimal cost to multiply Ai..Aj

\begingroup
\renewcommand\labelenumi{\theenumi:}
\begin{enumerate}
\item \textbf{function} NaiveMCM(p)  \label{item:1}
\item if i == j \label{item:2}
\item return 0 \label{item:3}
\item m[i,j] = infinity \label{item:4}
\item for k = i to j - 1 do \label{item:3}
\item \{ \label{item:4}
\item \indent q = NaiveMCM(p, i, k) + NaiveMCM(p, k+1, j) + PartialProducts\label{item:5}
\item \indent \} \label{item:7}
\item \indent if q < m[i,j] then \label{item:7}
\item \indent m[i,j] = q \label{item:8}
\item \indent return m[i,j] \label{item:8}
\end{enumerate}
\endgroup

\textbf{IterativeMCM}
\newline
Input: A sequence of matrix dimensions p
Output: The optimal cost to multiply A1..An

\begingroup
\renewcommand\labelenumi{\theenumi:}
\begin{enumerate}
\item \textbf{function} IterativeMCM(p)  \label{item:1}
\item for i = 1 to n do \label{item:2}
\item m[i,i] = 0 \label{item:3}
\item for len = 2 to n do \label{item:4}
\item \{ \label{item:4}
\item for i = 1 to n - len + 1 do \label{item:4}
\item \{ \label{item:4}
\item j = i + len - 1 \label{item:4}
\item m[i,j] = infinity \label{item:4}
\item for k = i to j - 1 do\label{item:4}
\item q = m[i,k] + m[k+1,j] + PartialProducts \label{item:4}
\item if q < m[i,j] then \label{item:4}
\item m[i,j] = q \label{item:4}
\item \} \label{item:4}
\item \} \label{item:4}
\item return m[1,n] \label{item:4}
\end{enumerate}
\endgroup

\textbf{MemoizationMCM}
\newline
Input: A sequence of matrix dimensions p
Output: The optimal cost to multiply A1..An

\begingroup
\renewcommand\labelenumi{\theenumi:}
\begin{enumerate}
\item \textbf{function} MemoizationMCM(p)  \label{item:1}
\item return MemoizationMCM(p, 1, n) \label{item:4}
\item \textbf{function} MCMHelper(p, i, j)  \label{item:1}
\item \{ \label{item:4}
\item if m[i,j] == infinity \label{item:4}
\item if i == j then
\item m[i,j] = 0 \label{item:4}\label{item:4}
\item else \label{item:4}
\item \{ \label{item:4}
\item for k = i to j - 1 do\label{item:4}
\item left = MCMHelper(p, i, k) \label{item:4}
\item right = MCMHelper(p, k+1, j \label{item:4}
\item total = left + right + PartialProducts \label{item:4}
\item if total < m[i,j] then \label{item:4}
\item m[i,j] = total \label{item:4}
\item return m[i,j] \label{item:4}
\item \} \label{item:4}
\item \} \label{item:4}
\item return m[1,n] \label{item:4}
\end{enumerate}
\endgroup

\section{Algorithm}
\begin{verbatim}



























{ Function to multiply two matrices (Naive }
function NaiveMCM(A: Matrix; RA, CA: integer; 
B: Matrix; RB, CB: integer; var Result: Matrix): boolean;
var
  i, j, k: integer;
begin
  if CA <> RB then
  begin
    writeln('Matrix multiplication not possible');
    NaiceMCM := false;
    exit;
  end;

  { Initialize result matrix }
  for i := 1 to RA do
    for j := 1 to CB do
      Result[i, j] := 0;

  { Perform multiplication }
  for i := 1 to RA do
    for j := 1 to CB do
      for k := 1 to CA do
        Result[i, j] := Result[i, j] + (A[i, k] * B[k, j]);

  NaiveMCM := true;
end;

function IterativeMCM(n: integer): LongInt;
begin
  { Initialize diagonal elements to 0, as cost of multiplying a single matrix is zero }
  for i := 1 to n do
    m[i, i] := 0;

  { len is the length of the subchain being considered (starts from 2 matrices up to n) }
  for len := 2 to n do
  begin
    for i := 1 to n - len + 1 do
    begin
      j := i + len - 1;
      m[i, j] := INF; { Initialize to a large value }

      for k := i to j - 1 do
      begin
        q := m[i, k] + m[k + 1, j] + (p[i - 1] * p[k] * p[j]);

        if q < m[i, j] then
          m[i, j] := q;
      end;
    end;
  end;

  { The minimum cost to multiply A1..An is stored in m[1, n] }
  IterativeMCM := m[1, n];
end;































































{ Recursive function with memoization to compute MCM }
function RecursiveMCM(i, j: integer): LongInt;
var
  k, q: integer;
begin
  if i = j then
    Exit(0); { Single matrix has zero multiplication cost }

  if m[i, j] <> -1 then
    Exit(m[i, j]); { Return already computed value }

  m[i, j] := INF; { Initialize to infinity }

  for k := i to j - 1 do
  begin
    q := RecursiveMCM(i, k) + RecursiveMCM(k + 1, j) + (p[i - 1] * p[k] * p[j]);
    
    if q < m[i, j] then
      m[i, j] := q;
  end;

  Exit(m[i, j]);
end;
\end{verbatim}


\section{Input examples}
If all matrixes had the same dimensions, or were already arranged optimally. Then the naive implementation and dynamic implementations would produce the same results. If the matrixes all had different dimensions and were not arranged optimally than dynamic programming would produce far better results. The left columns below represents input of matrices that would result in similar performance between algorithms, the column on the right represents input that would benefit from dynamic programming.
\begin{verbatim}
1 = 10x10               1 = 10x100
2 = 10x10               2 = 20x20
3 = 10x10               3 = 10x500
4 = 10x10               4 = 50x50
5 = 10x10               5 = 120x40
6 = 10x10               6 = 30x30
7 = 10x10               7 = 100x100
8 = 10x10               8 = 10x10
\end{verbatim}

\section{Time Complexity}
The time complexity of this algorithm is $theta(n^2)$. We get this result because will iterate n times, each iteration will consist of n-k steps. We are able to achieve this because we do not solve the same problems more than once.

\section{AI Tools Used}
ChatGPT v3.5 was used to generate the Pascal code. Algorithms used come from Pandurangan 2022.

\end{document}